\documentclass{article}
\usepackage[margin=1in]{geometry} 
\usepackage{amsmath,amsthm,amssymb,amsfonts, fancyhdr, color, comment, graphicx, environ}
\usepackage{xcolor}
\usepackage{amsmath, amssymb}
\usepackage{mdframed}
\usepackage[shortlabels]{enumitem}
\usepackage{indentfirst}
\usepackage{hyperref}
\renewcommand{\footrulewidth}{0.8 pt}
\hypersetup{
    colorlinks=true,
    linkcolor=blue,
    filecolor=magenta,      
    urlcolor=blue,
}


\pagestyle{fancy}



\newenvironment{problem}[2][Problem]
    { \begin{mdframed}[backgroundcolor=gray!20] \textbf{#1 #2} \\}
    {  \end{mdframed}}

    \fancyfoot[C]{} % Esto elimina la numeración de página en el pie de página
    

\newenvironment{solution}{\textbf{Solution}}


\lhead{}
\rhead{} 
\chead{\textbf{}}
% \lfoot{Alejandro Rodríguez Costello}
\rfoot{Facultad de Ciencias Exactas Ingeniería y Agrimensura}


\begin{document}
\title{\huge\bf Programación II\\[0.5cm]
        \bf\Large Trabajo práctico final: Texto Predictivo
        }
\author{\large\bf  Agustín López\\ \ \\}
\date{\large 
Facultad de Ciencias Exactas Ingeniería y Agrimensura}

\makeatletter
    \begin{titlepage}
        \begin{center}
	   { \includegraphics[width=10cm]{unr.png}}
	   {\ \\ \ \\}
        \vbox{}\vspace{2cm}
            {\@title}\\[3cm] 
            {\@author}
            % {\large Carrera: \bf Licenciatura en Ciencias de la Computación\\ \ \\}
            {\@date\\}

        \end{center}
    \end{titlepage}
\makeatother



\section{Programa C}
El objetivo principal del programa escrito en C se centró en dos tareas principales:

\begin{enumerate}
    \item \textbf{Leer archivos de la persona proporcionada como argumento:} \\
    En primer lugar, el programa ejecuta el comando \texttt{ls Textos<nombre>.txt}, almacenando su salida en un archivo auxiliar denominado
    \texttt{archivo.txt}. Esta acción tiene como fin conocer los nombres y la cantidad de textos escritos por la persona. Posteriormente, emplea un bucle while para leer línea por línea el archivo, procesando cada texto de manera individual en cada iteración.

    \item \textbf{Agregar la entrada sanitizada al archivo \texttt{Entradas/<nombre>.txt}:} \\
    En cada iteración del bucle mencionado, se ejecuta la función \texttt{agregar\_entrada}, la cual incorpora al archivo 
    \texttt{Entradas/<nombre>.txt} el contenido sanitizado del texto actual. Esta acción se realiza
     siguiendo las pautas indicadas en el enunciado del trabajo.
\end{enumerate}

Una vez finalizado el bucle, se invoca al programa escrito en Python, el cual se encuentra listo para operar
 con el archivo \texttt{Entradas/<nombre>.txt}.

En el programa en C, no hubo decisiones particularmente relevantes, ya que como el objetivo era bastante claro y directo, no logré encontrar muchas alternativas para abordarlo.

\section{Programa Python}
El programa en Python fue diseñado con el propósito principal de trabajar con tres archivos distintos: la entrada, las frases incompletas y la salida.

\begin{enumerate}
    \item \textbf{Carga inicial de datos:} \\
    La primera acción consistió en cargar las oraciones del archivo de entrada y las frases incompletas en dos listas de strings separadas.
    
    \item \textbf{Lógica central en la función \texttt{completar\_frases}:} \\
    Esta función posee la lógica central y la decisión más desafiante del programa: el algoritmo para determinar la palabra más probable. Después de investigar, encontré inicialmente el modelo
     de trigramas, basado en la creación de un diccionario donde las claves eran ternas de strings y los valores indicaban la frecuencia de ocurrencia en el archivo de entrada. Sin embargo, se identificó un
      problema al tratar la frase ``te \_ fumabas unos chinos en Madrid''. A pesar de encontrar múltiples instancias donde ``te'' era el primer elemento de la 3-upla, las palabras ``fumabas'', ``unos'', ``chinos'', ``en'' y ``Madrid'' no aparecían en el texto, lo que resultaba en un fallo en el modelo de trigramas.
    
    Para superar esta dificultad, adapté el modelo y se implementaron 2 duplas, lo que permitió una mejor aproximación al contexto. Al analizar las palabras posteriores a ``te'' y anteriores a ``fumabas'', se creó una estructura donde cada palabra candidata era una clave y el valor representaba la probabilidad de ser la palabra correcta. Finalmente, se seleccionó la palabra con el puntaje más alto para reemplazar el guion bajo en la frase incompleta.
\end{enumerate}

\end{document}